% Options for packages loaded elsewhere
\PassOptionsToPackage{unicode}{hyperref}
\PassOptionsToPackage{hyphens}{url}
%
\documentclass[
  11pt,
]{article}
\usepackage{amsmath,amssymb}
\usepackage{iftex}
\ifPDFTeX
  \usepackage[T1]{fontenc}
  \usepackage[utf8]{inputenc}
  \usepackage{textcomp} % provide euro and other symbols
\else % if luatex or xetex
  \usepackage{unicode-math} % this also loads fontspec
  \defaultfontfeatures{Scale=MatchLowercase}
  \defaultfontfeatures[\rmfamily]{Ligatures=TeX,Scale=1}
\fi
\usepackage{lmodern}
\ifPDFTeX\else
  % xetex/luatex font selection
\fi
% Use upquote if available, for straight quotes in verbatim environments
\IfFileExists{upquote.sty}{\usepackage{upquote}}{}
\IfFileExists{microtype.sty}{% use microtype if available
  \usepackage[]{microtype}
  \UseMicrotypeSet[protrusion]{basicmath} % disable protrusion for tt fonts
}{}
\makeatletter
\@ifundefined{KOMAClassName}{% if non-KOMA class
  \IfFileExists{parskip.sty}{%
    \usepackage{parskip}
  }{% else
    \setlength{\parindent}{0pt}
    \setlength{\parskip}{6pt plus 2pt minus 1pt}}
}{% if KOMA class
  \KOMAoptions{parskip=half}}
\makeatother
\usepackage{xcolor}
\usepackage[margin=1in]{geometry}
\usepackage{color}
\usepackage{fancyvrb}
\newcommand{\VerbBar}{|}
\newcommand{\VERB}{\Verb[commandchars=\\\{\}]}
\DefineVerbatimEnvironment{Highlighting}{Verbatim}{commandchars=\\\{\}}
% Add ',fontsize=\small' for more characters per line
\usepackage{framed}
\definecolor{shadecolor}{RGB}{248,248,248}
\newenvironment{Shaded}{\begin{snugshade}}{\end{snugshade}}
\newcommand{\AlertTok}[1]{\textcolor[rgb]{0.94,0.16,0.16}{#1}}
\newcommand{\AnnotationTok}[1]{\textcolor[rgb]{0.56,0.35,0.01}{\textbf{\textit{#1}}}}
\newcommand{\AttributeTok}[1]{\textcolor[rgb]{0.13,0.29,0.53}{#1}}
\newcommand{\BaseNTok}[1]{\textcolor[rgb]{0.00,0.00,0.81}{#1}}
\newcommand{\BuiltInTok}[1]{#1}
\newcommand{\CharTok}[1]{\textcolor[rgb]{0.31,0.60,0.02}{#1}}
\newcommand{\CommentTok}[1]{\textcolor[rgb]{0.56,0.35,0.01}{\textit{#1}}}
\newcommand{\CommentVarTok}[1]{\textcolor[rgb]{0.56,0.35,0.01}{\textbf{\textit{#1}}}}
\newcommand{\ConstantTok}[1]{\textcolor[rgb]{0.56,0.35,0.01}{#1}}
\newcommand{\ControlFlowTok}[1]{\textcolor[rgb]{0.13,0.29,0.53}{\textbf{#1}}}
\newcommand{\DataTypeTok}[1]{\textcolor[rgb]{0.13,0.29,0.53}{#1}}
\newcommand{\DecValTok}[1]{\textcolor[rgb]{0.00,0.00,0.81}{#1}}
\newcommand{\DocumentationTok}[1]{\textcolor[rgb]{0.56,0.35,0.01}{\textbf{\textit{#1}}}}
\newcommand{\ErrorTok}[1]{\textcolor[rgb]{0.64,0.00,0.00}{\textbf{#1}}}
\newcommand{\ExtensionTok}[1]{#1}
\newcommand{\FloatTok}[1]{\textcolor[rgb]{0.00,0.00,0.81}{#1}}
\newcommand{\FunctionTok}[1]{\textcolor[rgb]{0.13,0.29,0.53}{\textbf{#1}}}
\newcommand{\ImportTok}[1]{#1}
\newcommand{\InformationTok}[1]{\textcolor[rgb]{0.56,0.35,0.01}{\textbf{\textit{#1}}}}
\newcommand{\KeywordTok}[1]{\textcolor[rgb]{0.13,0.29,0.53}{\textbf{#1}}}
\newcommand{\NormalTok}[1]{#1}
\newcommand{\OperatorTok}[1]{\textcolor[rgb]{0.81,0.36,0.00}{\textbf{#1}}}
\newcommand{\OtherTok}[1]{\textcolor[rgb]{0.56,0.35,0.01}{#1}}
\newcommand{\PreprocessorTok}[1]{\textcolor[rgb]{0.56,0.35,0.01}{\textit{#1}}}
\newcommand{\RegionMarkerTok}[1]{#1}
\newcommand{\SpecialCharTok}[1]{\textcolor[rgb]{0.81,0.36,0.00}{\textbf{#1}}}
\newcommand{\SpecialStringTok}[1]{\textcolor[rgb]{0.31,0.60,0.02}{#1}}
\newcommand{\StringTok}[1]{\textcolor[rgb]{0.31,0.60,0.02}{#1}}
\newcommand{\VariableTok}[1]{\textcolor[rgb]{0.00,0.00,0.00}{#1}}
\newcommand{\VerbatimStringTok}[1]{\textcolor[rgb]{0.31,0.60,0.02}{#1}}
\newcommand{\WarningTok}[1]{\textcolor[rgb]{0.56,0.35,0.01}{\textbf{\textit{#1}}}}
\usepackage{graphicx}
\makeatletter
\newsavebox\pandoc@box
\newcommand*\pandocbounded[1]{% scales image to fit in text height/width
  \sbox\pandoc@box{#1}%
  \Gscale@div\@tempa{\textheight}{\dimexpr\ht\pandoc@box+\dp\pandoc@box\relax}%
  \Gscale@div\@tempb{\linewidth}{\wd\pandoc@box}%
  \ifdim\@tempb\p@<\@tempa\p@\let\@tempa\@tempb\fi% select the smaller of both
  \ifdim\@tempa\p@<\p@\scalebox{\@tempa}{\usebox\pandoc@box}%
  \else\usebox{\pandoc@box}%
  \fi%
}
% Set default figure placement to htbp
\def\fps@figure{htbp}
\makeatother
\setlength{\emergencystretch}{3em} % prevent overfull lines
\providecommand{\tightlist}{%
  \setlength{\itemsep}{0pt}\setlength{\parskip}{0pt}}
\setcounter{secnumdepth}{-\maxdimen} % remove section numbering
\usepackage{fancyvrb}
\usepackage{fvextra}
\fvset{breaklines=true, breakanywhere=true, fontsize=\small, frame=single}
\usepackage{amsmath}
\usepackage{amssymb}
\usepackage{fontspec}
\setmainfont{Times New Roman}
\usepackage{bookmark}
\IfFileExists{xurl.sty}{\usepackage{xurl}}{} % add URL line breaks if available
\urlstyle{same}
\hypersetup{
  pdftitle={Problem Set 4},
  pdfauthor={Ayush Pundyavana},
  hidelinks,
  pdfcreator={LaTeX via pandoc}}

\title{Problem Set 4}
\author{Ayush Pundyavana}
\date{}

\begin{document}
\maketitle

\begin{Shaded}
\begin{Highlighting}[]
\NormalTok{knitr}\SpecialCharTok{::}\NormalTok{opts\_chunk}\SpecialCharTok{$}\FunctionTok{set}\NormalTok{(}
\AttributeTok{echo =} \ConstantTok{TRUE}\NormalTok{,}
\AttributeTok{results =} \StringTok{\textquotesingle{}markup\textquotesingle{}}\NormalTok{,}
\AttributeTok{tidy =} \ConstantTok{TRUE}\NormalTok{,}
\AttributeTok{comment =} \ConstantTok{NA}\NormalTok{,}
\AttributeTok{width =} \DecValTok{60}\NormalTok{, }\CommentTok{\# wrap R output lines}
\AttributeTok{max.print =} \DecValTok{100} \CommentTok{\# limit huge outputs}
\NormalTok{)}
\FunctionTok{options}\NormalTok{(}\AttributeTok{width =} \DecValTok{60}\NormalTok{) }\CommentTok{\# ensures printed output wraps too}
\end{Highlighting}
\end{Shaded}

\begin{Shaded}
\begin{Highlighting}[]
\FunctionTok{library}\NormalTok{(dplyr)}
\end{Highlighting}
\end{Shaded}

\begin{verbatim}
Warning: package 'dplyr' was built under R version 4.4.3
\end{verbatim}

\begin{verbatim}

Attaching package: 'dplyr'
\end{verbatim}

\begin{verbatim}
The following objects are masked from 'package:stats':

    filter, lag
\end{verbatim}

\begin{verbatim}
The following objects are masked from 'package:base':

    intersect, setdiff, setequal, union
\end{verbatim}

\begin{Shaded}
\begin{Highlighting}[]
\FunctionTok{library}\NormalTok{(ggplot2)}
\end{Highlighting}
\end{Shaded}

\begin{verbatim}
Warning: package 'ggplot2' was built under R version 4.4.3
\end{verbatim}

\begin{Shaded}
\begin{Highlighting}[]
\FunctionTok{library}\NormalTok{(sandwich)}
\end{Highlighting}
\end{Shaded}

\begin{verbatim}
Warning: package 'sandwich' was built under R version 4.4.3
\end{verbatim}

\begin{Shaded}
\begin{Highlighting}[]
\FunctionTok{library}\NormalTok{(lmtest)}
\end{Highlighting}
\end{Shaded}

\begin{verbatim}
Warning: package 'lmtest' was built under R version 4.4.3
\end{verbatim}

\begin{verbatim}
Loading required package: zoo
\end{verbatim}

\begin{verbatim}
Warning: package 'zoo' was built under R version 4.4.3
\end{verbatim}

\begin{verbatim}

Attaching package: 'zoo'
\end{verbatim}

\begin{verbatim}
The following objects are masked from 'package:base':

    as.Date, as.Date.numeric
\end{verbatim}

\#Question 1

\begin{Shaded}
\begin{Highlighting}[]
\NormalTok{data1}\FloatTok{.1} \OtherTok{\textless{}{-}} \FunctionTok{read.csv}\NormalTok{(}\StringTok{"grades\_and\_temps.csv"}\NormalTok{)}

\NormalTok{data1}\FloatTok{.2} \OtherTok{\textless{}{-}}\NormalTok{ data1}\FloatTok{.1} \SpecialCharTok{\%\textgreater{}\%} \FunctionTok{mutate}\NormalTok{(}\AttributeTok{gdppc1k =} \FunctionTok{round}\NormalTok{(gdppc}\SpecialCharTok{/}\DecValTok{1000}\NormalTok{, }\DecValTok{3}\NormalTok{))}
\end{Highlighting}
\end{Shaded}

\begin{enumerate}
\def\labelenumi{\arabic{enumi}.}
\tightlist
\item
  Generate a scatter plot with GDP per capita in thousands of dollars
  (gdppc1k) on the x-axisand the average math score (math score) on the
  y-axis. Using visual inspection, do these variables seem to be
  positively correlated, negatively correlated, or not correlated at
  all? Does their relationship seem linear or non-linear?
\end{enumerate}

\begin{Shaded}
\begin{Highlighting}[]
\FunctionTok{plot}\NormalTok{(}\AttributeTok{x =}\NormalTok{ data1}\FloatTok{.2}\SpecialCharTok{$}\NormalTok{gdppc1k, }\AttributeTok{y =}\NormalTok{ data1}\FloatTok{.2}\SpecialCharTok{$}\NormalTok{math\_score,}
     \AttributeTok{main =} \StringTok{"Math Score based on GDP Per Capita "}\NormalTok{,}
     \AttributeTok{xlab =} \StringTok{"GDP Per Capita"}\NormalTok{,}
     \AttributeTok{ylab =} \StringTok{"Math Score"}\NormalTok{,}
     \AttributeTok{col =} \StringTok{"purple"}
\NormalTok{)}
\end{Highlighting}
\end{Shaded}

\pandocbounded{\includegraphics[keepaspectratio]{problem_set_4_files/figure-latex/unnamed-chunk-4-1.pdf}}

\begin{Shaded}
\begin{Highlighting}[]
\FunctionTok{cat}\NormalTok{(}\StringTok{"Using visual inspection of the scatterplot, the relationship between GDP}
\StringTok{    Per Capita and Math Score seems non{-}linear since the rate of change in the}
\StringTok{    Math score as GDP Per Capita increases seems to be non{-}constant."}\NormalTok{)}
\end{Highlighting}
\end{Shaded}

\begin{verbatim}
Using visual inspection of the scatterplot, the relationship between GDP
    Per Capita and Math Score seems non-linear since the rate of change in the
    Math score as GDP Per Capita increases seems to be non-constant.
\end{verbatim}

\begin{enumerate}
\def\labelenumi{\arabic{enumi}.}
\setcounter{enumi}{1}
\tightlist
\item
  Estimate the following two Equations:
\end{enumerate}

math scorei = a + b gdppc1ki + ei (1) math scorei = α + β1 gdppc1ki + β2
gdppc1k2i + εi (2)

Report the estimated coefficients ˆb, ˆβ1 and ˆβ2, along with their
heteroskedasticity-robust standard errors for each equation. Determine
whether the relationship between math scores and GDP per capita is
linear (as opposed to quadratic).

\begin{Shaded}
\begin{Highlighting}[]
\CommentTok{\#Linear model}
\NormalTok{lin\_model }\OtherTok{\textless{}{-}} \FunctionTok{lm}\NormalTok{(math\_score }\SpecialCharTok{\textasciitilde{}}\NormalTok{ gdppc1k, }\AttributeTok{data =}\NormalTok{ data1}\FloatTok{.2}\NormalTok{)}
\NormalTok{vcov\_m\_g }\OtherTok{\textless{}{-}} \FunctionTok{vcovHC}\NormalTok{(lin\_model, }\AttributeTok{type =} \StringTok{"HC3"}\NormalTok{)}
\FunctionTok{coeftest}\NormalTok{(lin\_model, }\AttributeTok{vcov. =}\NormalTok{ vcov\_m\_g)}
\end{Highlighting}
\end{Shaded}

\begin{verbatim}

t test of coefficients:

            Estimate Std. Error t value  Pr(>|t|)    
(Intercept) 432.9164     4.9696 87.1136 < 2.2e-16 ***
gdppc1k       1.4004     0.1621  8.6394 6.201e-16 ***
---
Signif. codes:  
0 '***' 0.001 '**' 0.01 '*' 0.05 '.' 0.1 ' ' 1
\end{verbatim}

\begin{Shaded}
\begin{Highlighting}[]
\NormalTok{b\_hat }\OtherTok{\textless{}{-}}\NormalTok{ lin\_model}\SpecialCharTok{$}\NormalTok{coefficients[}\DecValTok{2}\NormalTok{]}
\NormalTok{b\_hat\_SE }\OtherTok{\textless{}{-}} \FunctionTok{sqrt}\NormalTok{(}\FunctionTok{diag}\NormalTok{(vcov\_m\_g))[}\DecValTok{2}\NormalTok{]}

\FunctionTok{cat}\NormalTok{(}\StringTok{"b\_hat:"}\NormalTok{, b\_hat, }\StringTok{"}\SpecialCharTok{\textbackslash{}n}\StringTok{b\_hat SE:"}\NormalTok{, b\_hat\_SE)}
\end{Highlighting}
\end{Shaded}

\begin{verbatim}
b_hat: 1.400427 
b_hat SE: 0.1620984
\end{verbatim}

\begin{Shaded}
\begin{Highlighting}[]
\CommentTok{\#Quadratic model}
\NormalTok{quad\_model }\OtherTok{\textless{}{-}} \FunctionTok{lm}\NormalTok{(math\_score }\SpecialCharTok{\textasciitilde{}}\NormalTok{ gdppc1k }\SpecialCharTok{+} \FunctionTok{I}\NormalTok{(gdppc1k}\SpecialCharTok{*}\NormalTok{gdppc1k), }\AttributeTok{data =}\NormalTok{ data1}\FloatTok{.2}\NormalTok{)}
\NormalTok{vcov\_m\_g\_g2 }\OtherTok{\textless{}{-}} \FunctionTok{vcovHC}\NormalTok{(quad\_model, }\AttributeTok{type =} \StringTok{"HC3"}\NormalTok{)}
\FunctionTok{coeftest}\NormalTok{(quad\_model, }\AttributeTok{vcov. =}\NormalTok{ vcov\_m\_g\_g2)}
\end{Highlighting}
\end{Shaded}

\begin{verbatim}

t test of coefficients:

                        Estimate  Std. Error t value
(Intercept)          412.0714864   5.8861464 70.0070
gdppc1k                3.4619840   0.4068620  8.5090
I(gdppc1k * gdppc1k)  -0.0270478   0.0058909 -4.5915
                      Pr(>|t|)    
(Intercept)          < 2.2e-16 ***
gdppc1k              1.517e-15 ***
I(gdppc1k * gdppc1k) 6.917e-06 ***
---
Signif. codes:  
0 '***' 0.001 '**' 0.01 '*' 0.05 '.' 0.1 ' ' 1
\end{verbatim}

\begin{Shaded}
\begin{Highlighting}[]
\NormalTok{beta1\_hat }\OtherTok{\textless{}{-}}\NormalTok{ lin\_model}\SpecialCharTok{$}\NormalTok{coefficients[}\DecValTok{2}\NormalTok{]}
\NormalTok{beta1\_hat\_SE }\OtherTok{\textless{}{-}} \FunctionTok{sqrt}\NormalTok{(}\FunctionTok{diag}\NormalTok{(vcov\_m\_g))[}\DecValTok{2}\NormalTok{]}

\NormalTok{beta2\_hat }\OtherTok{\textless{}{-}}\NormalTok{ lin\_model}\SpecialCharTok{$}\NormalTok{coefficients[}\DecValTok{3}\NormalTok{]}
\NormalTok{beta2\_hat\_SE }\OtherTok{\textless{}{-}} \FunctionTok{sqrt}\NormalTok{(}\FunctionTok{diag}\NormalTok{(vcov\_m\_g))[}\DecValTok{3}\NormalTok{]}

\FunctionTok{cat}\NormalTok{(}\StringTok{"beta1\_hat:"}\NormalTok{, beta1\_hat, }\StringTok{"}\SpecialCharTok{\textbackslash{}n}\StringTok{beta1\_hat SE:"}\NormalTok{, beta1\_hat\_SE, }\StringTok{"}\SpecialCharTok{\textbackslash{}n\textbackslash{}n}\StringTok{beta2hat:"}\NormalTok{, beta2\_hat, }\StringTok{"}\SpecialCharTok{\textbackslash{}n}\StringTok{beta2\_hat SE:"}\NormalTok{, beta2\_hat\_SE, }\StringTok{"}\SpecialCharTok{\textbackslash{}n\textbackslash{}n}\StringTok{"}\NormalTok{)}
\end{Highlighting}
\end{Shaded}

\begin{verbatim}
beta1_hat: 1.400427 
beta1_hat SE: 0.1620984 

beta2hat: NA 
beta2_hat SE: NA 
\end{verbatim}

\begin{Shaded}
\begin{Highlighting}[]
\FunctionTok{cat}\NormalTok{(}\StringTok{"The relationship between math scores and GDP per capita is non{-}linear }
\StringTok{because beta2hat is statistically signficant"}\NormalTok{)}
\end{Highlighting}
\end{Shaded}

\begin{verbatim}
The relationship between math scores and GDP per capita is non-linear 
because beta2hat is statistically signficant
\end{verbatim}

\begin{enumerate}
\def\labelenumi{\arabic{enumi}.}
\setcounter{enumi}{2}
\tightlist
\item
  Using your estimated coefficients from Equation (2) above, what is the
  expected value of the difference in math scores between a country with
  a GDP per capita of \$5,000 and a country with a GDP per capita of
  \$10,000?
\end{enumerate}

\begin{Shaded}
\begin{Highlighting}[]
\NormalTok{pred\_5k }\OtherTok{\textless{}{-}} \FunctionTok{coef}\NormalTok{(quad\_model)[}\DecValTok{1}\NormalTok{] }\SpecialCharTok{+} \FunctionTok{coef}\NormalTok{(quad\_model)[}\DecValTok{2}\NormalTok{]}\SpecialCharTok{*}\DecValTok{5} \SpecialCharTok{+} \FunctionTok{coef}\NormalTok{(quad\_model)[}\DecValTok{3}\NormalTok{]}\SpecialCharTok{*}\DecValTok{5}\SpecialCharTok{\^{}}\DecValTok{2}
\NormalTok{pred\_10k }\OtherTok{\textless{}{-}} \FunctionTok{coef}\NormalTok{(quad\_model)[}\DecValTok{1}\NormalTok{] }\SpecialCharTok{+} \FunctionTok{coef}\NormalTok{(quad\_model)[}\DecValTok{2}\NormalTok{]}\SpecialCharTok{*}\DecValTok{10} \SpecialCharTok{+} \FunctionTok{coef}\NormalTok{(quad\_model)[}\DecValTok{3}\NormalTok{]}\SpecialCharTok{*}\DecValTok{10}\SpecialCharTok{\^{}}\DecValTok{2}
\NormalTok{diff\_pred }\OtherTok{\textless{}{-}}\NormalTok{ pred\_10k }\SpecialCharTok{{-}}\NormalTok{ pred\_5k}

\FunctionTok{cat}\NormalTok{(}\StringTok{"Expected difference in math scores (10k vs 5k GDP per capita):"}\NormalTok{,}
    \FunctionTok{round}\NormalTok{(diff\_pred, }\DecValTok{3}\NormalTok{))}
\end{Highlighting}
\end{Shaded}

\begin{verbatim}
Expected difference in math scores (10k vs 5k GDP per capita): 15.281
\end{verbatim}

\begin{enumerate}
\def\labelenumi{\arabic{enumi}.}
\setcounter{enumi}{3}
\tightlist
\item
  Generate a scatter plot with the math score on the y-axis and the
  (natural) logarithm of the GDP per capita (in thousands of dollars) on
  the x-axis. Based on visual inspection, is the relationship between
  these two transformed variables linear? Do you think that a linear-
  log specification will be able to better explain the relationship
  between these two variables compared to the variables in part 1?
\end{enumerate}

\begin{Shaded}
\begin{Highlighting}[]
\NormalTok{data1}\FloatTok{.3} \OtherTok{\textless{}{-}}\NormalTok{ data1}\FloatTok{.2} \SpecialCharTok{\%\textgreater{}\%} \FunctionTok{mutate}\NormalTok{(}\AttributeTok{log\_gdppc1k =} \FunctionTok{log}\NormalTok{(gdppc1k))}

\FunctionTok{plot}\NormalTok{(data1}\FloatTok{.3}\SpecialCharTok{$}\NormalTok{log\_gdppc1k, data1}\FloatTok{.2}\SpecialCharTok{$}\NormalTok{math\_score,}
     \AttributeTok{main =} \StringTok{"Math Score vs. log(GDP per capita)"}\NormalTok{,}
     \AttributeTok{xlab =} \StringTok{"log(GDP per capita in $1000s)"}\NormalTok{,}
     \AttributeTok{ylab =} \StringTok{"Math Score"}\NormalTok{,}
     \AttributeTok{col =} \StringTok{"darkgreen"}\NormalTok{, }\AttributeTok{pch =} \DecValTok{19}\NormalTok{)}
\end{Highlighting}
\end{Shaded}

\pandocbounded{\includegraphics[keepaspectratio]{problem_set_4_files/figure-latex/unnamed-chunk-8-1.pdf}}

\begin{Shaded}
\begin{Highlighting}[]
\FunctionTok{cat}\NormalTok{(}\StringTok{"Visual inspection shows a LINEAR relatinship after taking logs, indicating }
\StringTok{that the linear{-}log model might fit better than the raw{-}level model."}\NormalTok{)}
\end{Highlighting}
\end{Shaded}

\begin{verbatim}
Visual inspection shows a LINEAR relatinship after taking logs, indicating 
that the linear-log model might fit better than the raw-level model.
\end{verbatim}

\begin{enumerate}
\def\labelenumi{\arabic{enumi}.}
\setcounter{enumi}{4}
\tightlist
\item
  Estimate the following two regression equations:
\end{enumerate}

math scorei = θ1 + θ2 ln(gdppc1ki) + ui ln(math scorei) = γ1 + γ2
ln(gdppc1ki) + νi

Report the estimated intercept, slope, and their respective
heteroskedasticity-robust standard errors for each equation. How do you
interpret the intercept and the slope in each equation?

\begin{Shaded}
\begin{Highlighting}[]
\CommentTok{\#Linear{-}log model}
\NormalTok{linlog\_model }\OtherTok{\textless{}{-}} \FunctionTok{lm}\NormalTok{(math\_score }\SpecialCharTok{\textasciitilde{}}\NormalTok{ log\_gdppc1k, }\AttributeTok{data =}\NormalTok{ data1}\FloatTok{.3}\NormalTok{)}
\NormalTok{vcov\_li }\OtherTok{\textless{}{-}} \FunctionTok{vcovHC}\NormalTok{(linlog\_model, }\AttributeTok{type =} \StringTok{"HC3"}\NormalTok{)}
\FunctionTok{coeftest}\NormalTok{(linlog\_model, }\AttributeTok{vcov. =}\NormalTok{ vcov\_li)}
\end{Highlighting}
\end{Shaded}

\begin{verbatim}

t test of coefficients:

            Estimate Std. Error t value  Pr(>|t|)    
(Intercept) 374.3795     7.5512  49.579 < 2.2e-16 ***
log_gdppc1k  34.7885     2.5307  13.747 < 2.2e-16 ***
---
Signif. codes:  
0 '***' 0.001 '**' 0.01 '*' 0.05 '.' 0.1 ' ' 1
\end{verbatim}

\begin{Shaded}
\begin{Highlighting}[]
\CommentTok{\#Extract individual values}
\NormalTok{theta1\_hat  }\OtherTok{\textless{}{-}} \FunctionTok{coef}\NormalTok{(linlog\_model)[}\DecValTok{1}\NormalTok{]}
\NormalTok{theta2\_hat  }\OtherTok{\textless{}{-}} \FunctionTok{coef}\NormalTok{(linlog\_model)[}\DecValTok{2}\NormalTok{]}
\NormalTok{theta1\_se   }\OtherTok{\textless{}{-}} \FunctionTok{sqrt}\NormalTok{(}\FunctionTok{diag}\NormalTok{(}\FunctionTok{vcovHC}\NormalTok{(linlog\_model, }\AttributeTok{type =} \StringTok{"HC3"}\NormalTok{)))[}\DecValTok{1}\NormalTok{]}
\NormalTok{theta2\_se   }\OtherTok{\textless{}{-}} \FunctionTok{sqrt}\NormalTok{(}\FunctionTok{diag}\NormalTok{(}\FunctionTok{vcovHC}\NormalTok{(linlog\_model, }\AttributeTok{type =} \StringTok{"HC3"}\NormalTok{)))[}\DecValTok{2}\NormalTok{]}

\CommentTok{\#Report values}
\FunctionTok{cat}\NormalTok{(}\StringTok{"theta1hat (Intercept):"}\NormalTok{, theta1\_hat,}
\StringTok{"  theta1hatSE:"}\NormalTok{, theta1\_se,}\StringTok{"}\SpecialCharTok{\textbackslash{}n}\StringTok{"}\NormalTok{)}
\end{Highlighting}
\end{Shaded}

\begin{verbatim}
theta1hat (Intercept): 374.3795   theta1hatSE: 7.55121 
\end{verbatim}

\begin{Shaded}
\begin{Highlighting}[]
\FunctionTok{cat}\NormalTok{(}\StringTok{"theta2hat (Coefficient):"}\NormalTok{, theta2\_hat,}
\StringTok{"  theta2hatSE:"}\NormalTok{, theta2\_se,}\StringTok{"}\SpecialCharTok{\textbackslash{}n}\StringTok{"}\NormalTok{)}
\end{Highlighting}
\end{Shaded}

\begin{verbatim}
theta2hat (Coefficient): 34.78846   theta2hatSE: 2.530665 
\end{verbatim}

\begin{Shaded}
\begin{Highlighting}[]
\CommentTok{\#Log{-}log model}
\NormalTok{data1}\FloatTok{.4} \OtherTok{\textless{}{-}}\NormalTok{ data1}\FloatTok{.3} \SpecialCharTok{\%\textgreater{}\%} \FunctionTok{mutate}\NormalTok{(}\AttributeTok{log\_math =} \FunctionTok{log}\NormalTok{(math\_score))}
\NormalTok{loglog\_model }\OtherTok{\textless{}{-}} \FunctionTok{lm}\NormalTok{(log\_math }\SpecialCharTok{\textasciitilde{}}\NormalTok{ log\_gdppc1k, }\AttributeTok{data =}\NormalTok{ data1}\FloatTok{.4}\NormalTok{)}
\FunctionTok{coeftest}\NormalTok{(loglog\_model, }\AttributeTok{vcov =} \FunctionTok{vcovHC}\NormalTok{(loglog\_model, }\AttributeTok{type =} \StringTok{"HC3"}\NormalTok{))}
\end{Highlighting}
\end{Shaded}

\begin{verbatim}

t test of coefficients:

             Estimate Std. Error t value  Pr(>|t|)    
(Intercept) 5.9291352  0.0178519  332.13 < 2.2e-16 ***
log_gdppc1k 0.0785827  0.0059399   13.23 < 2.2e-16 ***
---
Signif. codes:  
0 '***' 0.001 '**' 0.01 '*' 0.05 '.' 0.1 ' ' 1
\end{verbatim}

\begin{Shaded}
\begin{Highlighting}[]
\CommentTok{\#Extract individual values neatly}
\NormalTok{gamma1\_hat  }\OtherTok{\textless{}{-}} \FunctionTok{coef}\NormalTok{(loglog\_model)[}\DecValTok{1}\NormalTok{]}
\NormalTok{gamma2\_hat  }\OtherTok{\textless{}{-}} \FunctionTok{coef}\NormalTok{(loglog\_model)[}\DecValTok{2}\NormalTok{]}
\NormalTok{gamma1\_se   }\OtherTok{\textless{}{-}} \FunctionTok{sqrt}\NormalTok{(}\FunctionTok{diag}\NormalTok{(}\FunctionTok{vcovHC}\NormalTok{(loglog\_model, }\AttributeTok{type =} \StringTok{"HC3"}\NormalTok{)))[}\DecValTok{1}\NormalTok{]}
\NormalTok{gamma2\_se   }\OtherTok{\textless{}{-}} \FunctionTok{sqrt}\NormalTok{(}\FunctionTok{diag}\NormalTok{(}\FunctionTok{vcovHC}\NormalTok{(loglog\_model, }\AttributeTok{type =} \StringTok{"HC3"}\NormalTok{)))[}\DecValTok{2}\NormalTok{]}

\CommentTok{\#Report values}
\FunctionTok{cat}\NormalTok{(}\StringTok{"gamma1hat (Intercept):"}\NormalTok{, gamma1\_hat,}
\StringTok{"  gamma1hatSE:"}\NormalTok{, gamma1\_se,}\StringTok{"}\SpecialCharTok{\textbackslash{}n}\StringTok{"}\NormalTok{)}
\end{Highlighting}
\end{Shaded}

\begin{verbatim}
gamma1hat (Intercept): 5.929135   gamma1hatSE: 0.01785189 
\end{verbatim}

\begin{Shaded}
\begin{Highlighting}[]
\FunctionTok{cat}\NormalTok{(}\StringTok{"gamma2hat (Coefficient):"}\NormalTok{, gamma2\_hat,}
\StringTok{"  gamma2hatSE:"}\NormalTok{, gamma2\_se,}\StringTok{"}\SpecialCharTok{\textbackslash{}n}\StringTok{"}\NormalTok{)}
\end{Highlighting}
\end{Shaded}

\begin{verbatim}
gamma2hat (Coefficient): 0.07858271   gamma2hatSE: 0.00593991 
\end{verbatim}

\begin{Shaded}
\begin{Highlighting}[]
\FunctionTok{cat}\NormalTok{(}\StringTok{"}\SpecialCharTok{\textbackslash{}n}\StringTok{Linear{-}log model Interpretations:}\SpecialCharTok{\textbackslash{}n\textbackslash{}n}\StringTok{"}\NormalTok{)}
\end{Highlighting}
\end{Shaded}

\begin{verbatim}

Linear-log model Interpretations:
\end{verbatim}

\begin{Shaded}
\begin{Highlighting}[]
\FunctionTok{cat}\NormalTok{(}\StringTok{"In the linear{-}log model, the slope (beta2hat) represents the change in }
\StringTok{math score for a 1\% change in GDP per capita.}\SpecialCharTok{\textbackslash{}n\textbackslash{}n}\StringTok{"}\NormalTok{)}
\end{Highlighting}
\end{Shaded}

\begin{verbatim}
In the linear-log model, the slope (beta2hat) represents the change in 
math score for a 1% change in GDP per capita.
\end{verbatim}

\begin{Shaded}
\begin{Highlighting}[]
\FunctionTok{cat}\NormalTok{(}\StringTok{"In the log{-}log model, the slope (gamma2hat) (elasticity) shows the \% change }
\StringTok{in math score for a 1\% change in GDP per capita."}\NormalTok{)}
\end{Highlighting}
\end{Shaded}

\begin{verbatim}
In the log-log model, the slope (gamma2hat) (elasticity) shows the % change 
in math score for a 1% change in GDP per capita.
\end{verbatim}

\begin{enumerate}
\def\labelenumi{\arabic{enumi}.}
\setcounter{enumi}{5}
\tightlist
\item
  Use the R-squared to determine which specification between the
  following pairs better explains the relationship between math scores
  and GDP per capita:
\end{enumerate}

\begin{enumerate}
\def\labelenumi{\alph{enumi})}
\tightlist
\item
  Quadratic vs Linear-log, b) Linear-log vs Log-log. Explain.
\end{enumerate}

\begin{Shaded}
\begin{Highlighting}[]
\CommentTok{\#Obtaining R\^{}2 values}
\NormalTok{R2\_lin }\OtherTok{\textless{}{-}} \FunctionTok{summary}\NormalTok{(lin\_model)}\SpecialCharTok{$}\NormalTok{r.squared}
\NormalTok{R2\_quad }\OtherTok{\textless{}{-}} \FunctionTok{summary}\NormalTok{(quad\_model)}\SpecialCharTok{$}\NormalTok{r.squared}
\NormalTok{R2\_linlog }\OtherTok{\textless{}{-}} \FunctionTok{summary}\NormalTok{(linlog\_model)}\SpecialCharTok{$}\NormalTok{r.squared}
\NormalTok{R2\_loglog }\OtherTok{\textless{}{-}} \FunctionTok{summary}\NormalTok{(loglog\_model)}\SpecialCharTok{$}\NormalTok{r.squared}

\FunctionTok{cat}\NormalTok{(}\StringTok{"R\_squared Linear:"}\NormalTok{, }\FunctionTok{round}\NormalTok{(R2\_lin, }\DecValTok{3}\NormalTok{),}
\StringTok{"}\SpecialCharTok{\textbackslash{}n}\StringTok{R\_squared Quadratic:"}\NormalTok{, }\FunctionTok{round}\NormalTok{(R2\_quad, }\DecValTok{3}\NormalTok{),}
\StringTok{"}\SpecialCharTok{\textbackslash{}n}\StringTok{R\_squared Linear{-}log:"}\NormalTok{, }\FunctionTok{round}\NormalTok{(R2\_linlog, }\DecValTok{3}\NormalTok{),}
\StringTok{"}\SpecialCharTok{\textbackslash{}n}\StringTok{R\_squared Log{-}log:"}\NormalTok{, }\FunctionTok{round}\NormalTok{(R2\_loglog, }\DecValTok{3}\NormalTok{))}
\end{Highlighting}
\end{Shaded}

\begin{verbatim}
R_squared Linear: 0.292 
R_squared Quadratic: 0.415 
R_squared Linear-log: 0.493 
R_squared Log-log: 0.492
\end{verbatim}

\begin{Shaded}
\begin{Highlighting}[]
\FunctionTok{cat}\NormalTok{(}\StringTok{"}\SpecialCharTok{\textbackslash{}n\textbackslash{}n}\StringTok{(a) Quadratic vs Linear{-}log: The model with higher R² explains more variance,}
\StringTok{but since they use different transformations, direct comparison must be cautious.}\SpecialCharTok{\textbackslash{}n\textbackslash{}n}\StringTok{"}\NormalTok{)}
\end{Highlighting}
\end{Shaded}

\begin{verbatim}


(a) Quadratic vs Linear-log: The model with higher R² explains more variance,
but since they use different transformations, direct comparison must be cautious.
\end{verbatim}

\begin{Shaded}
\begin{Highlighting}[]
\FunctionTok{cat}\NormalTok{(}\StringTok{"(b) Linear{-}log vs Log{-}log: Both have the same dependent variable transformation,}
\StringTok{so the higher R² model better fits the data."}\NormalTok{)}
\end{Highlighting}
\end{Shaded}

\begin{verbatim}
(b) Linear-log vs Log-log: Both have the same dependent variable transformation,
so the higher R² model better fits the data.
\end{verbatim}

\subsection{Question 2}\label{question-2}

\begin{Shaded}
\begin{Highlighting}[]
\NormalTok{data2}\FloatTok{.1} \OtherTok{\textless{}{-}} \FunctionTok{read.csv}\NormalTok{(}\StringTok{"DDCG\_dataset.csv"}\NormalTok{)}
\end{Highlighting}
\end{Shaded}

\begin{enumerate}
\def\labelenumi{\arabic{enumi}.}
\tightlist
\item
  For the sample of countries with available data in 2005, compute the
  mean GDP per capita for those that have a democracy (demo = 1). Do the
  same for the sample of countries that do not have a democracy in 2005
  (demo = 0). What is the difference? Can you reject the null hypothesis
  that the two means are equal at the 5\% significance level (no
  regressions needed for this part)? Estimate a regression of GDP per
  capita on the democracy dummy, also restricted to year 2005. How do
  the estimated intercept and slope compare to your previous
  calculations?
\end{enumerate}

\begin{Shaded}
\begin{Highlighting}[]
\CommentTok{\# Restrict to 2005}
\NormalTok{data2\_2005 }\OtherTok{\textless{}{-}}\NormalTok{ data2}\FloatTok{.1} \SpecialCharTok{\%\textgreater{}\%} \FunctionTok{filter}\NormalTok{(year }\SpecialCharTok{==} \DecValTok{2005}\NormalTok{)}

\CommentTok{\#Get group means}
\NormalTok{means }\OtherTok{\textless{}{-}}\NormalTok{ data2\_2005 }\SpecialCharTok{\%\textgreater{}\%}
  \FunctionTok{group\_by}\NormalTok{(dem) }\SpecialCharTok{\%\textgreater{}\%}
  \FunctionTok{summarise}\NormalTok{(}\AttributeTok{mean\_gdp =} \FunctionTok{mean}\NormalTok{(gdp\_capita, }\AttributeTok{na.rm =} \ConstantTok{TRUE}\NormalTok{))}

\NormalTok{means}
\end{Highlighting}
\end{Shaded}

\begin{verbatim}
# A tibble: 2 x 2
    dem mean_gdp
  <int>    <dbl>
1     0    4223.
2     1    9675.
\end{verbatim}

\begin{Shaded}
\begin{Highlighting}[]
\CommentTok{\#Difference in means}
\NormalTok{diff\_means }\OtherTok{\textless{}{-}}\NormalTok{ means}\SpecialCharTok{$}\NormalTok{mean\_gdp[means}\SpecialCharTok{$}\NormalTok{dem }\SpecialCharTok{==} \DecValTok{1}\NormalTok{] }\SpecialCharTok{{-}}\NormalTok{ means}\SpecialCharTok{$}\NormalTok{mean\_gdp[means}\SpecialCharTok{$}\NormalTok{dem }\SpecialCharTok{==} \DecValTok{0}\NormalTok{]}
\FunctionTok{cat}\NormalTok{(}\StringTok{"Difference in mean GDP per capita (democracy {-} non{-}democracy):"}\NormalTok{, }\FunctionTok{round}\NormalTok{(diff\_means, }\DecValTok{3}\NormalTok{), }\StringTok{"}\SpecialCharTok{\textbackslash{}n}\StringTok{"}\NormalTok{)}
\end{Highlighting}
\end{Shaded}

\begin{verbatim}
Difference in mean GDP per capita (democracy - non-democracy): 5452.018 
\end{verbatim}

\begin{Shaded}
\begin{Highlighting}[]
\CommentTok{\#Two{-}sample t{-}test (unequal variances)}
\NormalTok{t\_test }\OtherTok{\textless{}{-}} \FunctionTok{t.test}\NormalTok{(gdp\_capita }\SpecialCharTok{\textasciitilde{}}\NormalTok{ dem, }\AttributeTok{data =}\NormalTok{ data2\_2005, }\AttributeTok{var.equal =} \ConstantTok{FALSE}\NormalTok{)}
\NormalTok{t\_test}
\end{Highlighting}
\end{Shaded}

\begin{verbatim}

    Welch Two Sample t-test

data:  gdp_capita by dem
t = -3.1665, df = 120.04, p-value = 0.001956
alternative hypothesis: true difference in means between group 0 and group 1 is not equal to 0
95 percent confidence interval:
 -8860.969 -2043.067
sample estimates:
mean in group 0 mean in group 1 
       4222.954        9674.972 
\end{verbatim}

\begin{Shaded}
\begin{Highlighting}[]
\NormalTok{p\_val }\OtherTok{\textless{}{-}}\NormalTok{ t\_test}\SpecialCharTok{$}\NormalTok{p.value}

\FunctionTok{cat}\NormalTok{(}\StringTok{"}\SpecialCharTok{\textbackslash{}n}\StringTok{Interpretation: We reject the null hypothesis that mean GDP per capita is}
\StringTok{equal across statuses since  p{-}value("}\NormalTok{, }\FunctionTok{round}\NormalTok{(p\_val, }\DecValTok{3}\NormalTok{), }\StringTok{") is less than alpha("}\NormalTok{, }\FloatTok{0.05}\NormalTok{, }\StringTok{")}\SpecialCharTok{\textbackslash{}n}\StringTok{"}\NormalTok{)}
\end{Highlighting}
\end{Shaded}

\begin{verbatim}

Interpretation: We reject the null hypothesis that mean GDP per capita is
equal across statuses since  p-value( 0.002 ) is less than alpha( 0.05 )
\end{verbatim}

\begin{Shaded}
\begin{Highlighting}[]
\NormalTok{reg\_2005 }\OtherTok{\textless{}{-}} \FunctionTok{lm}\NormalTok{(gdp\_capita }\SpecialCharTok{\textasciitilde{}}\NormalTok{ dem, }\AttributeTok{data =}\NormalTok{ data2\_2005)}
\FunctionTok{coeftest}\NormalTok{(reg\_2005, }\AttributeTok{vcov =} \FunctionTok{vcovHC}\NormalTok{(reg\_2005, }\AttributeTok{type =} \StringTok{"HC3"}\NormalTok{))}
\end{Highlighting}
\end{Shaded}

\begin{verbatim}

t test of coefficients:

            Estimate Std. Error t value  Pr(>|t|)    
(Intercept)   4223.0     1188.2  3.5540 0.0005271 ***
dem           5452.0     1736.3  3.1399 0.0020856 ** 
---
Signif. codes:  
0 '***' 0.001 '**' 0.01 '*' 0.05 '.' 0.1 ' ' 1
\end{verbatim}

\begin{Shaded}
\begin{Highlighting}[]
\FunctionTok{cat}\NormalTok{(}\StringTok{"Intercept (alpha) should equal mean(GDP | dem=0):"}\NormalTok{, }
    \FunctionTok{round}\NormalTok{(means}\SpecialCharTok{$}\NormalTok{mean\_gdp[means}\SpecialCharTok{$}\NormalTok{dem }\SpecialCharTok{==} \DecValTok{0}\NormalTok{], }\DecValTok{3}\NormalTok{), }\StringTok{"}\SpecialCharTok{\textbackslash{}n}\StringTok{"}\NormalTok{)}
\end{Highlighting}
\end{Shaded}

\begin{verbatim}
Intercept (alpha) should equal mean(GDP | dem=0): 4222.954 
\end{verbatim}

\begin{Shaded}
\begin{Highlighting}[]
\FunctionTok{cat}\NormalTok{(}\StringTok{"Slope (Beta\_hat) should equal difference in means:"}\NormalTok{, }
    \FunctionTok{round}\NormalTok{(means}\SpecialCharTok{$}\NormalTok{mean\_gdp[means}\SpecialCharTok{$}\NormalTok{dem }\SpecialCharTok{==} \DecValTok{1}\NormalTok{] }\SpecialCharTok{{-}}\NormalTok{ means}\SpecialCharTok{$}\NormalTok{mean\_gdp[means}\SpecialCharTok{$}\NormalTok{dem }\SpecialCharTok{==} \DecValTok{0}\NormalTok{], }\DecValTok{3}\NormalTok{), }\StringTok{"}\SpecialCharTok{\textbackslash{}n}\StringTok{"}\NormalTok{)}
\end{Highlighting}
\end{Shaded}

\begin{verbatim}
Slope (Beta_hat) should equal difference in means: 5452.018 
\end{verbatim}

\begin{enumerate}
\def\labelenumi{\arabic{enumi}.}
\setcounter{enumi}{1}
\tightlist
\item
  Throughout the rest of this question, we are going to focus on
  studying the rela- tionship between the natural logarithm of GDP per
  capita and the democracy dummy:
\end{enumerate}

log(gdp capita) = α + βdem + ε

Estimate the equation above. Report the estimated coefficients ˆα and
ˆβ, along with their corresponding heteroskedasticity-robust standard
errors. Interpret ˆβ. Is β statistically different from zero at the 5\%
significance level (report t-statistic and p-value)?

\begin{Shaded}
\begin{Highlighting}[]
\CommentTok{\# Add log of GDP per capita}
\NormalTok{data2}\FloatTok{.2} \OtherTok{\textless{}{-}}\NormalTok{ data2}\FloatTok{.1} \SpecialCharTok{\%\textgreater{}\%}
  \FunctionTok{mutate}\NormalTok{(}\AttributeTok{log\_gdp =} \FunctionTok{log}\NormalTok{(gdp\_capita))}

\NormalTok{model3 }\OtherTok{\textless{}{-}} \FunctionTok{lm}\NormalTok{(log\_gdp }\SpecialCharTok{\textasciitilde{}}\NormalTok{ dem, }\AttributeTok{data =}\NormalTok{ data2}\FloatTok{.2}\NormalTok{)}
\NormalTok{vcov }\OtherTok{\textless{}{-}} \FunctionTok{vcovHC}\NormalTok{(model3, }\AttributeTok{type =} \StringTok{"HC3"}\NormalTok{)}
\FunctionTok{coeftest}\NormalTok{(model3, }\AttributeTok{vcov. =}\NormalTok{ vcov)}
\end{Highlighting}
\end{Shaded}

\begin{verbatim}

t test of coefficients:

            Estimate Std. Error t value  Pr(>|t|)    
(Intercept)  7.10971    0.12621 56.3312 < 2.2e-16 ***
dem          1.06652    0.16046  6.6468 1.033e-10 ***
---
Signif. codes:  
0 '***' 0.001 '**' 0.01 '*' 0.05 '.' 0.1 ' ' 1
\end{verbatim}

\begin{Shaded}
\begin{Highlighting}[]
\CommentTok{\# Extract estimates}
\NormalTok{alpha\_hat }\OtherTok{\textless{}{-}} \FunctionTok{coef}\NormalTok{(model3)[}\DecValTok{1}\NormalTok{]}
\NormalTok{beta\_hat  }\OtherTok{\textless{}{-}} \FunctionTok{coef}\NormalTok{(model3)[}\DecValTok{2}\NormalTok{]}
\NormalTok{se\_alpha  }\OtherTok{\textless{}{-}} \FunctionTok{sqrt}\NormalTok{(}\FunctionTok{diag}\NormalTok{(}\FunctionTok{vcovHC}\NormalTok{(model3, }\AttributeTok{type =} \StringTok{"HC3"}\NormalTok{)))[}\DecValTok{1}\NormalTok{]}
\NormalTok{se\_beta   }\OtherTok{\textless{}{-}} \FunctionTok{sqrt}\NormalTok{(}\FunctionTok{diag}\NormalTok{(}\FunctionTok{vcovHC}\NormalTok{(model3, }\AttributeTok{type =} \StringTok{"HC3"}\NormalTok{)))[}\DecValTok{2}\NormalTok{]}

\NormalTok{t\_beta }\OtherTok{\textless{}{-}}\NormalTok{ beta\_hat }\SpecialCharTok{/}\NormalTok{ se\_beta}

\FunctionTok{cat}\NormalTok{(}\StringTok{"alpha\_hat ="}\NormalTok{, }\FunctionTok{round}\NormalTok{(alpha\_hat, }\DecValTok{3}\NormalTok{), }\StringTok{" (SE ="}\NormalTok{, }\FunctionTok{round}\NormalTok{(se\_alpha, }\DecValTok{3}\NormalTok{), }\StringTok{")}\SpecialCharTok{\textbackslash{}n}\StringTok{"}\NormalTok{,}
\StringTok{"Beta\_hat ="}\NormalTok{, }\FunctionTok{round}\NormalTok{(beta\_hat, }\DecValTok{3}\NormalTok{), }\StringTok{" (SE ="}\NormalTok{, }\FunctionTok{round}\NormalTok{(se\_beta, }\DecValTok{3}\NormalTok{), }\StringTok{", t ="}\NormalTok{, }\FunctionTok{round}\NormalTok{(t\_beta, }\DecValTok{3}\NormalTok{), }\StringTok{")}\SpecialCharTok{\textbackslash{}n}\StringTok{"}\NormalTok{)}
\end{Highlighting}
\end{Shaded}

\begin{verbatim}
alpha_hat = 7.11  (SE = 0.126 )
 Beta_hat = 1.067  (SE = 0.16 , t = 6.647 )
\end{verbatim}

\begin{Shaded}
\begin{Highlighting}[]
\FunctionTok{cat}\NormalTok{(}\StringTok{"}\SpecialCharTok{\textbackslash{}n}\StringTok{Interpretation: Beta\_hat represents the percentage difference in GDP per capita}
\StringTok{between democracies and non{-}democracies. If Beta\_hat is statistically significant}
\StringTok{(|t| \textgreater{} 2), democracies have higher average log(GDP per capita)."}\NormalTok{)}
\end{Highlighting}
\end{Shaded}

\begin{verbatim}

Interpretation: Beta_hat represents the percentage difference in GDP per capita
between democracies and non-democracies. If Beta_hat is statistically significant
(|t| > 2), democracies have higher average log(GDP per capita).
\end{verbatim}

\begin{enumerate}
\def\labelenumi{\arabic{enumi}.}
\setcounter{enumi}{2}
\tightlist
\item
  Do you think that the equation above can suffer from omitted variable
  bias (OVB)? List two omitted variables that can potentially lead to
  OVB when estimating Equation (5). For each of them, argue whether the
  estimate ˆβ is upward or downward biased (i.e., would ˆβ go down or
  up, if we included the omitted variable)?
\end{enumerate}

\begin{Shaded}
\begin{Highlighting}[]
\FunctionTok{cat}\NormalTok{(}\StringTok{"Possible ommitted variables are education levels (more educated populations}
\StringTok{{-}{-}\textgreater{} higher GDP and higher likelihood of democracy), trade openness (trade boosts}
\StringTok{income and correlates with democratic institutions)"}\NormalTok{)}
\end{Highlighting}
\end{Shaded}

\begin{verbatim}
Possible ommitted variables are education levels (more educated populations
--> higher GDP and higher likelihood of democracy), trade openness (trade boosts
income and correlates with democratic institutions)
\end{verbatim}

\begin{Shaded}
\begin{Highlighting}[]
\FunctionTok{cat}\NormalTok{(}\StringTok{"}\SpecialCharTok{\textbackslash{}n\textbackslash{}n}\StringTok{If these are ommitted, and both correlate positively with democracy and }
\StringTok{GDP, then beta1hat is upward biased, meaning the democracy effect appears }
\StringTok{larger than it really is."}\NormalTok{)}
\end{Highlighting}
\end{Shaded}

\begin{verbatim}


If these are ommitted, and both correlate positively with democracy and 
GDP, then beta1hat is upward biased, meaning the democracy effect appears 
larger than it really is.
\end{verbatim}

\begin{enumerate}
\def\labelenumi{\arabic{enumi}.}
\setcounter{enumi}{3}
\tightlist
\item
\end{enumerate}

\begin{Shaded}
\begin{Highlighting}[]
\NormalTok{model4 }\OtherTok{\textless{}{-}} \FunctionTok{lm}\NormalTok{(log\_gdp }\SpecialCharTok{\textasciitilde{}}\NormalTok{ dem }\SpecialCharTok{+}\NormalTok{ lp\_bl }\SpecialCharTok{+}\NormalTok{ lh\_bl, }\AttributeTok{data =}\NormalTok{ data2}\FloatTok{.2}\NormalTok{)}
\FunctionTok{coeftest}\NormalTok{(model4, }\AttributeTok{vcov =} \FunctionTok{vcovHC}\NormalTok{(model4, }\AttributeTok{type =} \StringTok{"HC3"}\NormalTok{))}
\end{Highlighting}
\end{Shaded}

\begin{verbatim}

t test of coefficients:

             Estimate Std. Error t value Pr(>|t|)    
(Intercept) 6.1858112  0.1833171 33.7438  < 2e-16 ***
dem         0.3822453  0.1651491  2.3145  0.02117 *  
lp_bl       0.0082106  0.0037405  2.1951  0.02876 *  
lh_bl       0.0942623  0.0102318  9.2127  < 2e-16 ***
---
Signif. codes:  
0 '***' 0.001 '**' 0.01 '*' 0.05 '.' 0.1 ' ' 1
\end{verbatim}

\begin{Shaded}
\begin{Highlighting}[]
\CommentTok{\# Extract Beta\_hat and SE}
\NormalTok{beta\_hat\_edu }\OtherTok{\textless{}{-}} \FunctionTok{coef}\NormalTok{(model4)[}\StringTok{"dem"}\NormalTok{]}
\NormalTok{beta\_se\_edu  }\OtherTok{\textless{}{-}} \FunctionTok{sqrt}\NormalTok{(}\FunctionTok{diag}\NormalTok{(}\FunctionTok{vcovHC}\NormalTok{(model4, }\AttributeTok{type =} \StringTok{"HC3"}\NormalTok{)))[}\DecValTok{2}\NormalTok{]}

\FunctionTok{cat}\NormalTok{(}\StringTok{"With education controls:}\SpecialCharTok{\textbackslash{}n}\StringTok{Beta\_hat ="}\NormalTok{, }\FunctionTok{round}\NormalTok{(beta\_hat\_edu, }\DecValTok{3}\NormalTok{),}
    \StringTok{"| SE ="}\NormalTok{, }\FunctionTok{round}\NormalTok{(beta\_se\_edu, }\DecValTok{3}\NormalTok{), }\StringTok{"}\SpecialCharTok{\textbackslash{}n}\StringTok{"}\NormalTok{)}
\end{Highlighting}
\end{Shaded}

\begin{verbatim}
With education controls:
Beta_hat = 0.382 | SE = 0.165 
\end{verbatim}

\begin{Shaded}
\begin{Highlighting}[]
\FunctionTok{cat}\NormalTok{(}\StringTok{"}\SpecialCharTok{\textbackslash{}n}\StringTok{If Beta\_hat decreases after adding education controls, the earlier model }
\StringTok{likely overstated democracy\textquotesingle{}s effect, meaning the initial estimate was upward }
\StringTok{biased."}\NormalTok{)}
\end{Highlighting}
\end{Shaded}

\begin{verbatim}

If Beta_hat decreases after adding education controls, the earlier model 
likely overstated democracy's effect, meaning the initial estimate was upward 
biased.
\end{verbatim}

\begin{enumerate}
\def\labelenumi{\arabic{enumi}.}
\setcounter{enumi}{4}
\tightlist
\item
  On top of lp bl and lh bl, also add ginv and tradewb as control
  variables in Equation (5). How does ˆβ change (report the standard
  error as well)? Provide an intuitive explanation. Does this change
  mean that the ˆβ was upward or downward biased in part 4?
\end{enumerate}

\begin{Shaded}
\begin{Highlighting}[]
\NormalTok{model5 }\OtherTok{\textless{}{-}} \FunctionTok{lm}\NormalTok{(log\_gdp }\SpecialCharTok{\textasciitilde{}}\NormalTok{ dem }\SpecialCharTok{+}\NormalTok{ lp\_bl }\SpecialCharTok{+}\NormalTok{ lh\_bl }\SpecialCharTok{+}\NormalTok{ ginv }\SpecialCharTok{+}\NormalTok{ tradewb, }\AttributeTok{data =}\NormalTok{ data2}\FloatTok{.2}\NormalTok{)}
\FunctionTok{coeftest}\NormalTok{(model5, }\AttributeTok{vcov =} \FunctionTok{vcovHC}\NormalTok{(model5, }\AttributeTok{type =} \StringTok{"HC3"}\NormalTok{))}
\end{Highlighting}
\end{Shaded}

\begin{verbatim}

t test of coefficients:

               Estimate  Std. Error t value  Pr(>|t|)    
(Intercept)  5.73386870  0.28315581 20.2499 < 2.2e-16 ***
dem          0.43158752  0.15213068  2.8370  0.004799 ** 
lp_bl        0.00806050  0.00375963  2.1440  0.032672 *  
lh_bl        0.09224219  0.00954171  9.6673 < 2.2e-16 ***
ginv        -0.00753655  0.00877757 -0.8586  0.391096    
tradewb      0.00711982  0.00099077  7.1862 3.566e-12 ***
---
Signif. codes:  
0 '***' 0.001 '**' 0.01 '*' 0.05 '.' 0.1 ' ' 1
\end{verbatim}

\begin{Shaded}
\begin{Highlighting}[]
\CommentTok{\# Extract updated Beta\_hat and SE}
\NormalTok{beta\_hat\_econ }\OtherTok{\textless{}{-}} \FunctionTok{coef}\NormalTok{(model5)[}\StringTok{"dem"}\NormalTok{]}
\NormalTok{beta\_se\_econ  }\OtherTok{\textless{}{-}} \FunctionTok{sqrt}\NormalTok{(}\FunctionTok{diag}\NormalTok{(}\FunctionTok{vcovHC}\NormalTok{(model5, }\AttributeTok{type =} \StringTok{"HC3"}\NormalTok{)))[}\DecValTok{2}\NormalTok{]}

\FunctionTok{cat}\NormalTok{(}\StringTok{"With education + economic controls:}\SpecialCharTok{\textbackslash{}n}\StringTok{Beta\_hat ="}\NormalTok{, }\FunctionTok{round}\NormalTok{(beta\_hat\_econ, }\DecValTok{3}\NormalTok{),}
    \StringTok{"| SE ="}\NormalTok{, }\FunctionTok{round}\NormalTok{(beta\_se\_econ, }\DecValTok{3}\NormalTok{), }\StringTok{"}\SpecialCharTok{\textbackslash{}n}\StringTok{"}\NormalTok{)}
\end{Highlighting}
\end{Shaded}

\begin{verbatim}
With education + economic controls:
Beta_hat = 0.432 | SE = 0.152 
\end{verbatim}

\begin{Shaded}
\begin{Highlighting}[]
\CommentTok{\#compute changes}
\NormalTok{beta\_change }\OtherTok{\textless{}{-}}\NormalTok{ beta\_hat\_econ }\SpecialCharTok{{-}}\NormalTok{ beta\_hat\_edu}
\NormalTok{percent\_change }\OtherTok{\textless{}{-}} \DecValTok{100} \SpecialCharTok{*}\NormalTok{ beta\_change }\SpecialCharTok{/}\NormalTok{ beta\_hat\_edu}

\ControlFlowTok{if}\NormalTok{ (beta\_change }\SpecialCharTok{\textless{}} \DecValTok{0}\NormalTok{) \{}
  \FunctionTok{cat}\NormalTok{(}\StringTok{"}\SpecialCharTok{\textbackslash{}n\textbackslash{}n}\StringTok{Interpretation:}\SpecialCharTok{\textbackslash{}n\textbackslash{}n}\StringTok{"}\NormalTok{,}
      \StringTok{"After adding ginv and tradewb, the democracy coefficient Beta\_hat decreased.}\SpecialCharTok{\textbackslash{}n}\StringTok{"}\NormalTok{,}
      \StringTok{"This indicates that the positive association between democracy and income was partly explained}\SpecialCharTok{\textbackslash{}n}\StringTok{"}\NormalTok{,}
      \StringTok{"by higher investment and trade levels typical of democratic countries.}\SpecialCharTok{\textbackslash{}n}\StringTok{"}\NormalTok{,}
      \StringTok{"Therefore, the Beta\_hat from part 4 was **upward biased** — omitting ginv and tradewb made democracy appear}\SpecialCharTok{\textbackslash{}n}\StringTok{"}\NormalTok{,}
      \StringTok{"to have a stronger effect on GDP per capita than it truly does once those factors are controlled for.}\SpecialCharTok{\textbackslash{}n}\StringTok{"}\NormalTok{)}
\NormalTok{\} }\ControlFlowTok{else} \ControlFlowTok{if}\NormalTok{ (beta\_change }\SpecialCharTok{\textgreater{}} \DecValTok{0}\NormalTok{) \{}
  \FunctionTok{cat}\NormalTok{(}\StringTok{"}\SpecialCharTok{\textbackslash{}n\textbackslash{}n}\StringTok{Interpretation:}\SpecialCharTok{\textbackslash{}n\textbackslash{}n}\StringTok{"}\NormalTok{,}
      \StringTok{"After adding ginv and tradewb, the democracy coefficient Beta\_hat }
\StringTok{increase. This suggests that trade and investment were negatively correlated }
\StringTok{with democracy or positively correlated with GDP in a way that suppressed }
\StringTok{democracy’s apparent effect. Thus, the Beta\_hat from part 4 was downward biased.}
\StringTok{Omitting ginv and tradewb understated democracy’s association with GDP per }
\StringTok{capita.}\SpecialCharTok{\textbackslash{}n}\StringTok{"}\NormalTok{)\}}
\end{Highlighting}
\end{Shaded}

\begin{verbatim}


Interpretation:

 After adding ginv and tradewb, the democracy coefficient Beta_hat 
increase. This suggests that trade and investment were negatively correlated 
with democracy or positively correlated with GDP in a way that suppressed 
democracy’s apparent effect. Thus, the Beta_hat from part 4 was downward biased.
Omitting ginv and tradewb understated democracy’s association with GDP per 
capita.
\end{verbatim}

\begin{enumerate}
\def\labelenumi{\arabic{enumi}.}
\setcounter{enumi}{5}
\tightlist
\item
  Using the estimates from part 5, can we reject the null hypothesis
  that the two control variables related to the economy are irrelevant
  in predicting GDP per capita (i.e., thecoefficients on both tradewb
  and ginv are zero)?
\end{enumerate}

\begin{Shaded}
\begin{Highlighting}[]
\CommentTok{\#install.packages("car")}
\FunctionTok{library}\NormalTok{(car)}
\end{Highlighting}
\end{Shaded}

\begin{verbatim}
Warning: package 'car' was built under R version 4.4.3
\end{verbatim}

\begin{verbatim}
Loading required package: carData
\end{verbatim}

\begin{verbatim}
Warning: package 'carData' was built under R version 4.4.3
\end{verbatim}

\begin{verbatim}

Attaching package: 'car'
\end{verbatim}

\begin{verbatim}
The following object is masked from 'package:dplyr':

    recode
\end{verbatim}

\begin{Shaded}
\begin{Highlighting}[]
\CommentTok{\# Perform robust Wald test (heteroskedasticity{-}robust F{-}test)}
\NormalTok{joint\_test }\OtherTok{\textless{}{-}} \FunctionTok{linearHypothesis}\NormalTok{(model5, }
                               \FunctionTok{c}\NormalTok{(}\StringTok{"ginv = 0"}\NormalTok{, }\StringTok{"tradewb = 0"}\NormalTok{), }
                               \AttributeTok{vcov =} \FunctionTok{vcovHC}\NormalTok{(model5, }\AttributeTok{type =} \StringTok{"HC3"}\NormalTok{))}

\NormalTok{joint\_test}
\end{Highlighting}
\end{Shaded}

\begin{verbatim}

Linear hypothesis test:
ginv = 0
tradewb = 0

Model 1: restricted model
Model 2: log_gdp ~ dem + lp_bl + lh_bl + ginv + tradewb

Note: Coefficient covariance matrix supplied.

  Res.Df Df      F    Pr(>F)    
1    381                        
2    379  2 25.821 3.069e-11 ***
---
Signif. codes:  
0 '***' 0.001 '**' 0.01 '*' 0.05 '.' 0.1 ' ' 1
\end{verbatim}

\begin{Shaded}
\begin{Highlighting}[]
\CommentTok{\# {-}{-}{-} Interpretation block {-}{-}{-}}
\NormalTok{p\_value }\OtherTok{\textless{}{-}}\NormalTok{ joint\_test[}\DecValTok{2}\NormalTok{, }\StringTok{"Pr(\textgreater{}F)"}\NormalTok{]}

\ControlFlowTok{if}\NormalTok{ (p\_value }\SpecialCharTok{\textless{}} \FloatTok{0.05}\NormalTok{) \{}
  \FunctionTok{cat}\NormalTok{(}\StringTok{"}\SpecialCharTok{\textbackslash{}n}\StringTok{Result:}\SpecialCharTok{\textbackslash{}n\textbackslash{}n}\StringTok{"}\NormalTok{,}
      \StringTok{"We REJECT the null hypothesis (p{-}value ="}\NormalTok{, }\FunctionTok{round}\NormalTok{(p\_value, }\DecValTok{4}\NormalTok{), }\StringTok{"). This }
\StringTok{means ginv and tradewb are jointly significant predictors of }
\StringTok{log(GDP per capita). Economic factors like investment and trade openness help }
\StringTok{explain cross{-}country differences in income levels.}\SpecialCharTok{\textbackslash{}n}\StringTok{"}\NormalTok{)}
\NormalTok{\} }\ControlFlowTok{else}\NormalTok{ \{}
  \FunctionTok{cat}\NormalTok{(}\StringTok{"}\SpecialCharTok{\textbackslash{}n}\StringTok{Result:}\SpecialCharTok{\textbackslash{}n}\StringTok{"}\NormalTok{,}
      \StringTok{"We FAIL TO REJECT the null hypothesis (p{-}value ="}\NormalTok{, }\FunctionTok{round}\NormalTok{(p\_value, }\DecValTok{4}\NormalTok{), }\StringTok{").}\SpecialCharTok{\textbackslash{}n}\StringTok{"}\NormalTok{,}
      \StringTok{"This means there is no statistical evidence at the 5\% level that ginv and tradewb are jointly relevant.}\SpecialCharTok{\textbackslash{}n}\StringTok{"}\NormalTok{,}
      \StringTok{"Adding them does not significantly improve the model’s ability to predict GDP per capita.}\SpecialCharTok{\textbackslash{}n}\StringTok{"}\NormalTok{)}
\NormalTok{\}}
\end{Highlighting}
\end{Shaded}

\begin{verbatim}

Result:

 We REJECT the null hypothesis (p-value = 0 ). This 
means ginv and tradewb are jointly significant predictors of 
log(GDP per capita). Economic factors like investment and trade openness help 
explain cross-country differences in income levels.
\end{verbatim}

\end{document}
